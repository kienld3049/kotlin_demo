\documentclass[12pt,a4paper,oneside]{book} % twoside for draf
%\newcommand{\neuronText}{nơ-ron\xspace}
\newcommand{\vecto}{véc-tơ\xspace}
\newcommand{\deeplearning}{học sâu\xspace} 
\newcommand{\bias}{\textit{bias}\xspace}
\newcommand{\variance}{\textit{variance}}
\newcommand{\tool}{AdvGeneration\xspace}

\newcommand{\model}{M\xspace}
\newcommand{\weights}{\textbf{W}}
\newcommand{\parameters}{\bm{\theta}}
\newcommand{\lr}{\eta}
\newcommand{\weightElement}{w_{i,j,k}}
\newcommand{\weightBetweenTwoLayer}[2]{\textbf{w}_{#1,#2}}
\newcommand{\numOfLayer}{l}
\newcommand{\activations}[0]{\bm{\theta}}




\newcommand{\tieuchi}{$||L||_2$\xspace}
\newcommand{\proposedMethod}{\textit{AE4DNN}\xspace}
\newcommand{\crossEntropy}{\textit{cross-entropy}\xspace}
\newcommand{\reRank}{\textit{re-rank}\xspace}


\newcommand{\neuron}[2]{n_{#1}^{#2}}
\newcommand{\dataset}{\textbf{X}}
\newcommand{\numFeatures}{d\xspace}
\newcommand{\numClasses}{k\xspace}
\newcommand{\numSamples}{s\xspace}
\newcommand{\subsetX}{\textbf{S}}


\newcommand{\outputOfDeepModel}[1]{$F(#1)$}

% input vector
\newcommand{\inputVector}{\textbf{x}\xspace}
\newcommand{\originLabel}{y^{true}_\inputVector\xspace}
\newcommand{\inputVectorWithIndex}[1]{\inputVector_{#1}}
\newcommand{\classOfInputVector}{c_{\inputVector}}
\newcommand{\predictedProbOfInputVector}{\textbf{y}}
\newcommand{\trueProbOfInputVector}{\textbf{y}^{true}}
\newcommand{\trueProbOfInputVectorForAll}{\textbf{Y}^{true}}
% \newcommand{\orignLabelSet}{\textbf{Y}^{true}}
% adversary
\newcommand{\adversaryVector}{\textbf{x}'}
\newcommand{\adversaryVectorForAll}{\textbf{X}'}
\newcommand{\adversaryVectorWithIndex}[1]{\textbf{x}'_{#1}}
\newcommand{\classOfAdversaryVector}{c'}
\newcommand{\predictedProbOfAdversaryVector}{\textbf{y}'}
\newcommand{\configPair}{($\beta, \phi$)\xspace}
\newcommand{\adversarialSet}{\textbf{X}'}

% target
\newcommand{\targetLabel}{y^*}
\newcommand{\targetProb}{\textbf{y}^*}

% objective function
\newcommand{\originalObjective}{f(\inputVector, \weights, \activations)}
\newcommand{\originalObjectiveForAll}{f(\dataset, \weights, \trueProbOfInputVectorForAll)}
\newcommand{\originalObjectiveForAllVHai}{J(\dataset, \weights, \trueProbOfInputVectorForAll)}
% gradient of objective function
\newcommand{\gradientOriginalObjective}{\nabla_{\inputVector} f(\inputVector, \weights, \activations)}
\newcommand{\gradientOriginalObjectiveByW}{\nabla_{\weights} f(\inputVector, \weights, \activations)}
\newcommand{\gradientOriginalObjectiveByWeightElement}{\frac{\partial{\originalObjective}}{\partial{\weightElement}}}
\newcommand{\gradientTargetObjective}{\nabla_{\inputVector} f(\inputVector, \weights, \activations)}

\newcommand{\gradientOriginalObjectiveForAll}{\nabla_{\inputVector} \originalObjectiveForAll}



% fgsm relate
\newcommand{\gradientUntargetedFGSM}{\nabla_{\inputVector} J(\inputVector,  \trueProbOfInputVector_{\inputVector})}

\newcommand{\gradienttargetedFGSM}{\nabla_{\inputVector} J(\inputVector,  \targetLabel)}

% carlini relate

\newcommand{\carnili}{Carlini-Wagner \tieuchi\xspace}
\newcommand{\Carnili}{Carlini-Wagner \tieuchi\xspace}
% L_BFGS
\newcommand{\LBFGS}{Box-constrained L-BFGS\xspace}
\newcommand{\lbfgsLoss}{J(\adversaryVector, \targetLabel)}
\newcommand{\lbfgs}{Box-constrained L-BFGS\xspace}
% Iterative Least-Likely Class

\newcommand{\leastlikely}{\textit{Iterative Least-Likely Class}\xspace}
\newcommand{\leastlikelyshort}{l.l. class\xspace}
\newcommand{\gradientLLClass}{\nabla_{\inputVector} J(\adversaryVector_{N-1}, \targetLabel)}


% L

\newcommand{\LKhong}{$||L||_0$\xspace}
\newcommand{\LVoCung}{$||L||_{\infty}$\xspace}
\newcommand{\LP}{$||L||_p$\xspace}



% exp

\newcommand{\mSecure}{\model_{secure}\xspace}




%\usepackage{babel}
\usepackage[utf8]{vietnam}


\usepackage{mathptmx}	% same Time New Roma
\usepackage{amsmath}

\usepackage{fancyhdr}
% \usepackage[utf8]{inputenc}
\usepackage[vietnamese]{babel}
\usepackage{titlesec}
\usepackage{titletoc}
\usepackage{listings}
\usepackage[unicode,bookmarks=true]{hyperref}
\usepackage{bookmark}
\usepackage[left=3cm,right=2cm,top=2.5cm,bottom=3cm]{geometry}
\usepackage{graphicx}
\usepackage{tikz}
\usepackage{varwidth}
\usepackage{float}
\usepackage{color}
\usepackage{multirow}
\usepackage{booktabs}
\usepackage[linesnumbered,lined,ruled,resetcount, algochapter]{algorithm2e}
\usepackage{svg}
\usepackage{tabularx}
\usepackage{nomencl}
\usepackage{scrfontsizes}
\usepackage{longtable}
\usepackage{multicol}
\usepackage[toc,page]{appendix}
\usepackage{adjustbox}
\usepackage{natbib}
\usepackage{makecell}
\usepackage{longtable}
\usepackage{multirow}
% \usepackage{algpseudocode}
\usepackage{setting/bkthesis}
\usepackage[toc]{appendix}
\usepackage{bm}

\usepackage{dirtytalk}

\usepackage{subcaption}

\counterwithin{figure}{chapter}
% \usepackage[
% backend=biber,
% bibstyle=authoryear,
% sorting=none,
% citestyle=numeric-comp
% ]{biblatex}
\usepackage{xpatch}

\makeatletter
% \input{numeric.bbx}
\makeatother




\setlength{\parskip}{6pt}

\usetikzlibrary{calc}
\setlength{\parindent}{10mm}
\renewcommand{\baselinestretch}{1.31}
\graphicspath{{images/}}
\renewcommand{\listalgorithmcfname}{Danh sách thuật toán}
\renewcommand{\nomname}{Danh sách ký hiệu, viết tắt}
\renewcommand\appendixtocname{Phụ lục}



\newtheorem{definition}{Định nghĩa}
\SetKwRepeat{Do}{do}{while}%

\titlecontents{chapter}%
[0pt]%
{\vspace{1ex}}%
{\bfseries Chương \thecontentslabel\quad}%
{\bfseries}%
{\bfseries\hfill\contentspage}


\let\chapappname\chaptername


\definecolor{dkgreen}{rgb}{0,0.6,0}
\definecolor{gray}{rgb}{0.5,0.5,0.5}
\definecolor{mauve}{rgb}{0.58,0,0.82}

% Định nghĩa highlighting cho Kotlin
\lstdefinelanguage{Kotlin}{
  keywords={package, as, typealias, this, super, val, var, fun, for, null, true, false, is, in, throw, return, break, continue, object, if, try, else, while, do, when, interface, class, enum, object, override, public, private, internal, protected, catch, finally, out, ref, vararg, by, constructor, init, companion, lateinit, data, inline, noinline, tailrec, external, annotation, crossinline, const, operator, infix, suspend, sealed, abstract, open, final},
  ndkeywords={@Deprecated, @JvmName, @JvmStatic, @JvmOverloads, @JvmField, @Throws, Iterable, Int, Integer, Short, Byte, Long, Double, Float, String, Runnable, Array, List, Map, Set, Collection, StringBuilder},
  sensitive=true,
  comment=[l]{//},
  morecomment=[s]{/*}{*/},
  morestring=[b]",
  morestring=[s]{"""*}{*"""},
}

\lstset{
  language=Kotlin,
  basicstyle=\ttfamily\small,
  keywordstyle=\color{blue}\bfseries,
  ndkeywordstyle=\color{mauve},
  commentstyle=\color{dkgreen}\itshape,
  stringstyle=\color{red},
  showstringspaces=false,
  breaklines=true,
  frame=single,
  numbers=left,
  numberstyle=\tiny\color{gray},
  captionpos=b,
  tabsize=2
}

\renewcommand{\lstlistingname}{Source}
\renewcommand\thechapter{\arabic{chapter}}
\renewcommand\thesection{\thechapter.\arabic{section}}
\renewcommand\thesubsection{\thesection.\arabic{subsection}}
% \renewcommand\thesubsubsection{\thesection.\thesubsection.\arabic{subsubsection}}
\renewcommand{\thetable}{\thechapter.\arabic{table}}
\renewcommand{\thefigure}{\thechapter.\arabic{figure}}
\renewcommand{\thealgocf}{\thechapter.\arabic{algocf}}
\renewcommand{\thedefinition}{\thechapter.\arabic{definition}}


\begin{document}

\renewcommand{\thelstlisting}{\thechapter.\arabic{lstlisting}}


\pagestyle{plain}
\frontmatter

%-------TITLE PAGE------%
\begin{titlepage}
	\center
	\begin{tikzpicture}[overlay,remember picture]
		\draw [line width=3pt,rounded corners=0pt,]
		($ (current page.north west) + (25mm,-25mm) $)
		rectangle
		($ (current page.south east) + (-15mm,25mm) $);
		\draw [line width=1pt,rounded corners=0pt]
		($ (current page.north west) + (26.5mm,-26.5mm) $)
		rectangle
		($ (current page.south east) + (-16.5mm,26.5mm) $);
	\end{tikzpicture}
	
	{\large \bfseries ĐẠI HỌC QUỐC GIA HÀ NỘI\\ TRƯỜNG ĐẠI HỌC CÔNG NGHỆ}\\[1cm]
	\includegraphics[width=0.25\linewidth]{images/Logo_UET.png}\\[1cm]
	{\Large  \bfseries  NHÓM 9}\\[1.5cm]
	
	{ \LARGE \bfseries  TÌM HIỂU VỀ NGÔN NGỮ KOTLIN}\\[0.05cm]
    
	\hfill\\[2cm]
	{\large \bfseries BÁO CÁO MÔN NGUYÊN LÝ CÁC NGÔN NGỮ LẬP TRÌNH}\\	
	\vspace{7mm}
	{\large \bfseries Ngành: Khoa học máy tính}	
	\hfill\\[5.3cm]	
	{\large \bfseries HÀ NỘI - 2023}\\	
	\vfill
\end{titlepage}

%-------TITLE PAGE+6hbk,------%
\begin{titlepage}
	\center
	\begin{tikzpicture}[overlay,remember picture]
	\draw [line width=3pt,rounded corners=0pt,]
	($ (current page.north west) + (25mm,-25mm) $)
	rectangle
	($ (current page.south east) + (-15mm,25mm) $);
	\draw [line width=1pt,rounded corners=0pt]
	($ (current page.north west) + (26.5mm,-26.5mm) $)
	rectangle
	($ (current page.south east) + (-16.5mm,26.5mm) $);
	\end{tikzpicture}
	
	{\large \bfseries ĐẠI HỌC QUỐC GIA HÀ NỘI\\ TRƯỜNG ĐẠI HỌC CÔNG NGHỆ}\\[2cm]
	
	{\Large  \bfseries  Lê Đức Kiên}\\[2cm]
		{ \LARGE \bfseries TRÍ TUỆ NHÂN TẠO - NHỮNG LỢI ÍCH VÀ MỐI NGUY HẠI TỚI LOÀI NGƯỜI}\\[0.05cm]
	\hfill\\[1.5cm]
	{\large \bfseries BÁO CÁO MÔN CÁC VẤN ĐỀ HIỆN ĐẠI TRONG KHMT}\\	
	\vspace{7mm}
	{\large \bfseries Ngành: Khoa học máy tính}
	\hfill\\[2cm]
	\begin{flushleft}
	    	{\large \bfseries Cán bộ hướng dẫn: GS.TS. Nguyễn Thanh Thuỷ}\\
	    		% {\large \bfseries \hspace{4.2cm}  TS. Trần Hoàng Việt}\\
	\end{flushleft}
	
	\hfill\\[1.5cm]	
	\begin{flushleft}
	\end{flushleft}
		\hfill\\[2.4cm]	
	{\large \bfseries HÀ NỘI - 2023}\\		
	\vfill		
\end{titlepage}

%-------TITLE PAGE+6hbk,------%

\newcommand{\neuronText}{nơ-ron\xspace}
\newcommand{\vecto}{véc-tơ\xspace}
\newcommand{\deeplearning}{học sâu\xspace} 
\newcommand{\bias}{\textit{bias}\xspace}
\newcommand{\variance}{\textit{variance}}
\newcommand{\tool}{AdvGeneration\xspace}

\newcommand{\model}{M\xspace}
\newcommand{\weights}{\textbf{W}}
\newcommand{\parameters}{\bm{\theta}}
\newcommand{\lr}{\eta}
\newcommand{\weightElement}{w_{i,j,k}}
\newcommand{\weightBetweenTwoLayer}[2]{\textbf{w}_{#1,#2}}
\newcommand{\numOfLayer}{l}
\newcommand{\activations}[0]{\bm{\theta}}




\newcommand{\tieuchi}{$||L||_2$\xspace}
\newcommand{\proposedMethod}{\textit{AE4DNN}\xspace}
\newcommand{\crossEntropy}{\textit{cross-entropy}\xspace}
\newcommand{\reRank}{\textit{re-rank}\xspace}


\newcommand{\neuron}[2]{n_{#1}^{#2}}
\newcommand{\dataset}{\textbf{X}}
\newcommand{\numFeatures}{d\xspace}
\newcommand{\numClasses}{k\xspace}
\newcommand{\numSamples}{s\xspace}
\newcommand{\subsetX}{\textbf{S}}


\newcommand{\outputOfDeepModel}[1]{$F(#1)$}

% input vector
\newcommand{\inputVector}{\textbf{x}\xspace}
\newcommand{\originLabel}{y^{true}_\inputVector\xspace}
\newcommand{\inputVectorWithIndex}[1]{\inputVector_{#1}}
\newcommand{\classOfInputVector}{c_{\inputVector}}
\newcommand{\predictedProbOfInputVector}{\textbf{y}}
\newcommand{\trueProbOfInputVector}{\textbf{y}^{true}}
\newcommand{\trueProbOfInputVectorForAll}{\textbf{Y}^{true}}
% \newcommand{\orignLabelSet}{\textbf{Y}^{true}}
% adversary
\newcommand{\adversaryVector}{\textbf{x}'}
\newcommand{\adversaryVectorForAll}{\textbf{X}'}
\newcommand{\adversaryVectorWithIndex}[1]{\textbf{x}'_{#1}}
\newcommand{\classOfAdversaryVector}{c'}
\newcommand{\predictedProbOfAdversaryVector}{\textbf{y}'}
\newcommand{\configPair}{($\beta, \phi$)\xspace}
\newcommand{\adversarialSet}{\textbf{X}'}

% target
\newcommand{\targetLabel}{y^*}
\newcommand{\targetProb}{\textbf{y}^*}

% objective function
\newcommand{\originalObjective}{f(\inputVector, \weights, \activations)}
\newcommand{\originalObjectiveForAll}{f(\dataset, \weights, \trueProbOfInputVectorForAll)}
\newcommand{\originalObjectiveForAllVHai}{J(\dataset, \weights, \trueProbOfInputVectorForAll)}
% gradient of objective function
\newcommand{\gradientOriginalObjective}{\nabla_{\inputVector} f(\inputVector, \weights, \activations)}
\newcommand{\gradientOriginalObjectiveByW}{\nabla_{\weights} f(\inputVector, \weights, \activations)}
\newcommand{\gradientOriginalObjectiveByWeightElement}{\frac{\partial{\originalObjective}}{\partial{\weightElement}}}
\newcommand{\gradientTargetObjective}{\nabla_{\inputVector} f(\inputVector, \weights, \activations)}

\newcommand{\gradientOriginalObjectiveForAll}{\nabla_{\inputVector} \originalObjectiveForAll}



% fgsm relate
\newcommand{\gradientUntargetedFGSM}{\nabla_{\inputVector} J(\inputVector,  \trueProbOfInputVector_{\inputVector})}

\newcommand{\gradienttargetedFGSM}{\nabla_{\inputVector} J(\inputVector,  \targetLabel)}

% carlini relate

\newcommand{\carnili}{Carlini-Wagner \tieuchi\xspace}
\newcommand{\Carnili}{Carlini-Wagner \tieuchi\xspace}
% L_BFGS
\newcommand{\LBFGS}{Box-constrained L-BFGS\xspace}
\newcommand{\lbfgsLoss}{J(\adversaryVector, \targetLabel)}
\newcommand{\lbfgs}{Box-constrained L-BFGS\xspace}
% Iterative Least-Likely Class

\newcommand{\leastlikely}{\textit{Iterative Least-Likely Class}\xspace}
\newcommand{\leastlikelyshort}{l.l. class\xspace}
\newcommand{\gradientLLClass}{\nabla_{\inputVector} J(\adversaryVector_{N-1}, \targetLabel)}


% L

\newcommand{\LKhong}{$||L||_0$\xspace}
\newcommand{\LVoCung}{$||L||_{\infty}$\xspace}
\newcommand{\LP}{$||L||_p$\xspace}



% exp

\newcommand{\mSecure}{\model_{secure}\xspace}








\tableofcontents

\newpage
\clearpage % Start a new page

\chapter*{Thuật ngữ}
% \fontsize{18pt}{11.0pt}\selectfont

\addcontentsline{toc}{chapter}{Thuật ngữ}

% \changefontsizes[16pt]{14pt}
% \begin{center} 
% 	\textbf{{THUẬT NGỮ}}
% \end{center}

\begin{table}[h!]

\resizebox{\textwidth}{!}{
\begin{tabular}{|l|l|l|}
\hline
\multicolumn{1}{|c|}{\textbf{Từ viết tắt}} & \multicolumn{1}{c|}{\textbf{Từ đầy đủ}}                                                 & \multicolumn{1}{c|}{\textbf{Ý nghĩa}}                                                                         \\ \hline
AI                               & Artificial Intelligence                                                           & Trí tuệ nhân tạo                                                                        \\ \hline
AGI                               &   Artificial General Intelligence                                                             & Trí tuệ tổng hợp nhân tạo                                                                        \\ \hline
ChatGPT                               &  Chat Generative Pre-training Transformer                                                             & một hệ thống trợ lý ảo dựa trên mô hình ngôn ngữ lớn                                                                        \\ \hline
GAN                               &  Generative Adversarial Networks                                                             & một Mạng đối nghịch tạo sinh                                                                        \\ \hline


\end{tabular}
}
\end{table}

\newpage
\clearpage % Start a new page

% \changefontsizes[16pt]{14pt}
% \addtocontents{toc}{\vspace{-1cm}}

% \chapter*{Lời cam đoan}
% \addcontentsline{toc}{chapter}{Lời cam đoan}
% % \begin{center}
% %     \textbf{LỜI CAM ĐOAN}
   
% % \end{center}

% Nội dung lời cam đoan

% \begin{flushright}
% 	\begin{varwidth}{\linewidth}\centering
% 		Hà Nội, ngày \space ... \space tháng \space ... \space năm ...\\
% 		\bigbreak
% 		Học viên\\[1.5cm]
% 		Nguyễn Văn A
% 	\end{varwidth}
% \end{flushright}

% \newpage
% \clearpage % Start a new page

% \chapter*{Lời cảm ơn}
% \addcontentsline{toc}{chapter}{Lời cảm ơn}
% % \begin{center}
% %     \textbf{LỜI CẢM ƠN}
% % \end{center}

% Nội dung lời cảm ơn 

% \begin{flushright}
% 	\begin{varwidth}{\linewidth}\centering
% 		Hà Nội, ngày \space ... \space tháng \space ... \space năm ...\\
% 		\bigbreak
% 		Sinh viên\\[1.5cm]
% 		Nguyễn Văn A
% 	\end{varwidth}
% \end{flushright}

\newpage
\clearpage % Start a new page

\chapter*{Tóm tắt}
\addcontentsline{toc}{chapter}{Tóm tắt}
% \begin{center}
%     \textbf{TÓM TẮT}
% \end{center}
\changefontsizes[16pt]{13pt}
% \textit{\textbf{Tóm tắt: }} 
% Tóm tắt nội dung của báo cáo (có thể có, có thể không)

\vspace{-0.5cm}
\begin{flushleft}
  \textit{\textbf{Từ khóa:} }
\end{flushleft}
% \changefontsizes[16pt]{13pt}


\newpage
\clearpage % Start a new page


% \chapter*{\listfigurename}
\listoffigures
\addcontentsline{toc}{chapter}{\listfigurename}


\newpage
\clearpage % Start a new page

% \makeatletter
% \renewcommand\listoftables{%
%         \@starttoc{lot}%
% }
% \makeatother

% \chapter*{\listtablename}
% \listoftables
% \addcontentsline{toc}{chapter}{\listtablename}


\mainmatter

\changefontsizes[16pt]{13pt}
\pagestyle{plain}

\chapter{Giới thiệu về Kotlin}

\section{Bối cảnh ra đời}

Kotlin là một ngôn ngữ lập trình hiện đại được phát triển bởi JetBrains, công ty nổi tiếng với các IDE như IntelliJ IDEA, PyCharm, và WebStorm. Dự án Kotlin bắt đầu vào năm 2010 và được công bố chính thức vào năm 2011. Phiên bản ổn định đầu tiên (1.0) được ra mắt vào tháng 2/2016.

Tên gọi "Kotlin" được đặt theo tên hòn đảo Kotlin gần St. Petersburg, Nga - nơi đặt văn phòng của JetBrains, tương tự như Java được đặt theo tên đảo Java của Indonesia.

\subsection{Động lực phát triển}

JetBrains đã phát triển Kotlin xuất phát từ những nhu cầu thực tế trong công việc của họ:

\begin{itemize}
\item \textbf{Tăng năng suất}: Cần một ngôn ngữ giúp lập trình viên viết mã nhanh hơn với ít boilerplate code hơn Java
\item \textbf{An toàn hơn}: Giảm thiểu các lỗi runtime phổ biến, đặc biệt là NullPointerException
\item \textbf{Tương thích với Java}: Cần khả năng tương tác 100\% với Java để tận dụng hệ sinh thái JVM
\item \textbf{Công cụ tốt hơn}: Là công ty phát triển IDE, JetBrains muốn một ngôn ngữ có hỗ trợ công cụ tuyệt vời ngay từ đầu
\end{itemize}

Theo Andrey Breslav - Lead Language Designer của Kotlin: "Chúng tôi muốn tăng năng suất của đội ngũ, và chúng tôi cần một ngôn ngữ tốt hơn Java nhưng vẫn tương thích với hệ sinh thái Java hiện có."

\section{Vị trí trong hệ sinh thái ngôn ngữ lập trình}

Kotlin được thiết kế để chạy trên Java Virtual Machine (JVM), nhưng không chỉ giới hạn ở đó. Ngôn ngữ này có thể được biên dịch thành:

\begin{itemize}
\item \textbf{JVM bytecode}: Chạy trên máy ảo Java
\item \textbf{JavaScript}: Cho phép phát triển ứng dụng web frontend
\item \textbf{Native code}: Thông qua Kotlin/Native, có thể biên dịch sang mã máy cho iOS, macOS, Windows, Linux
\end{itemize}

\subsection{So sánh với các ngôn ngữ khác}

\textbf{Kotlin vs Java}

Kotlin được xem là "Better Java" - một phiên bản cải tiến của Java với các ưu điểm:
\begin{itemize}
\item Cú pháp ngắn gọn hơn (ít boilerplate code)
\item Null safety được tích hợp vào type system
\item Extension functions
\item Coroutines cho lập trình bất đồng bộ
\item Smart casts
\end{itemize}

\textbf{Kotlin vs Scala}

Trong khi Scala tập trung nhiều vào lập trình hàm thuần túy và các khái niệm phức tạp, Kotlin theo đuổi triết lý thực dụng (pragmatic):
\begin{itemize}
\item Dễ học hơn Scala
\item Compilation time nhanh hơn
\item Tương thác Java tốt hơn
\item Nhưng ít mạnh mẽ hơn về khả năng FP so với Scala
\end{itemize}

\section{Mốc quan trọng trong lịch sử phát triển}

\begin{table}[h]
\centering
\begin{tabular}{|l|p{10cm}|}
\hline
\textbf{Năm} & \textbf{Sự kiện} \\
\hline
2010 & JetBrains bắt đầu dự án Kotlin \\
\hline
2011 & Công bố Kotlin ra công chúng \\
\hline
2016 & Phát hành Kotlin 1.0 (phiên bản ổn định đầu tiên) \\
\hline
2017 & Google công bố Kotlin là ngôn ngữ chính thức cho Android \\
\hline
2019 & Google tuyên bố Kotlin là ngôn ngữ ưu tiên (preferred language) cho Android \\
\hline
2021 & Kotlin trở thành một trong những ngôn ngữ phát triển nhanh nhất \\
\hline
\end{tabular}
\caption{Các mốc quan trọng trong lịch sử Kotlin}
\end{table}

Sự kiện quan trọng nhất là Google I/O 2017 khi Google công bố Kotlin là ngôn ngữ chính thức cho phát triển Android. Điều này đã thúc đẩy sự phát triển của Kotlin một cách đáng kể, với hàng triệu lập trình viên Android bắt đầu học và sử dụng Kotlin.

\section{Triết lý thiết kế}

Kotlin được xây dựng dựa trên bốn trụ cột chính:

\subsection{Pragmatic (Thực dụng)}

Kotlin không phải là ngôn ngữ nghiên cứu hay học thuật. Nó được thiết kế để giải quyết các vấn đề thực tế mà các nhà phát triển gặp phải hàng ngày:

\begin{itemize}
\item Tập trung vào production code, không phải prototype
\item Các tính năng được thêm vào dựa trên nhu cầu thực tế
\item Không theo đuổi sự hoàn hảo về mặt lý thuyết nếu nó làm phức tạp việc sử dụng
\end{itemize}

\subsection{Concise (Ngắn gọn)}

Một trong những mục tiêu chính là giảm thiểu boilerplate code:

\textbf{Ví dụ: Data class}

Trong Java:
\begin{lstlisting}[language=Java]
public class Person {
    private String name;
    private int age;
    
    public Person(String name, int age) {
        this.name = name;
        this.age = age;
    }
    
    public String getName() { return name; }
    public int getAge() { return age; }
    
    @Override
    public boolean equals(Object o) { ... }
    
    @Override
    public int hashCode() { ... }
    
    @Override
    public String toString() { ... }
}
\end{lstlisting}

Trong Kotlin:
\begin{lstlisting}[language=Kotlin]
data class Person(val name: String, val age: Int)
\end{lstlisting}

Chỉ với một dòng code, Kotlin tự động tạo ra constructor, getters, \texttt{equals()}, \texttt{hashCode()}, \texttt{toString()}, và \texttt{copy()}.

\subsection{Safe (An toàn)}

Kotlin tích hợp nhiều cơ chế an toàn ngay trong ngôn ngữ:

\begin{itemize}
\item \textbf{Null Safety}: Phân biệt rõ ràng giữa nullable và non-nullable types
\item \textbf{No checked exceptions}: Loại bỏ các checked exception gây rườm rà
\item \textbf{Smart casts}: Tự động ép kiểu sau khi kiểm tra
\end{itemize}

\subsection{Interoperable (Khả năng tương tác)}

Kotlin được thiết kế để tương tác 100\% với Java:

\begin{itemize}
\item Có thể gọi code Java từ Kotlin và ngược lại
\item Sử dụng các thư viện Java hiện có
\item Có thể mix Java và Kotlin trong cùng một project
\item Biên dịch thành JVM bytecode giống như Java
\end{itemize}

Điều này cho phép các team áp dụng Kotlin từng bước (incremental adoption) thay vì phải rewrite toàn bộ codebase.

\section{Ứng dụng và cộng đồng}

\subsection{Lĩnh vực ứng dụng}

Kotlin hiện được sử dụng rộng rãi trong nhiều lĩnh vực:

\begin{itemize}
\item \textbf{Android Development}: Ứng dụng chính và phổ biến nhất
\item \textbf{Backend Development}: Spring Boot, Ktor framework
\item \textbf{Web Frontend}: Kotlin/JS
\item \textbf{Multiplatform}: Kotlin Multiplatform Mobile (KMM) cho iOS và Android
\item \textbf{Data Science}: Kotlin for Data Science
\end{itemize}

\subsection{Thống kê và xu hướng}

Theo các nguồn thống kê:

\begin{itemize}
\item Hơn 60\% ứng dụng Android top 1000 sử dụng Kotlin (2021)
\item Hơn 95\% các ứng dụng mới trên Android sử dụng Kotlin (2022)
\item Stack Overflow Developer Survey liên tục xếp Kotlin trong top "Most Loved Languages"
\item GitHub Octoverse xếp Kotlin trong top ngôn ngữ phát triển nhanh nhất
\end{itemize}

\subsection{Các công ty lớn sử dụng Kotlin}

\begin{itemize}
\item Google (Android, nhiều dự án nội bộ)
\item Netflix (backend services)
\item Pinterest (mobile app)
\item Uber (internal tools)
\item Trello (mobile app)
\item Evernote (mobile app)
\item Slack (mobile app)
\end{itemize}

\section{Tổng quan về báo cáo}

Báo cáo này sẽ phân tích Kotlin từ góc độ nguyên lý ngôn ngữ lập trình (Programming Language Principles), tập trung vào các khía cạnh:

\begin{itemize}
\item \textbf{Phần I - Cơ sở lý thuyết}: Triết lý thiết kế, hệ thống kiểu, và null safety
\item \textbf{Phần II - Paradigms \& Features}: Lập trình hàm, quản lý trạng thái, xử lý lỗi, đệ quy, và generics
\item \textbf{Phần III - Advanced \& Practical}: Extension functions, DSLs, coroutines, và Java interop
\end{itemize}

Mỗi chương sẽ phân tích cả góc độ lý thuyết (từ "The Joy of Kotlin") và thực tế (từ "Kotlin in Action"), giúp người đọc hiểu sâu về các quyết định thiết kế và ứng dụng thực tế của Kotlin.

\chapter{Triết lý thiết kế và Nguyên lý cốt lõi}

[Nội dung Chapter 2 sẽ được viết tiếp...]

\chapter{Hệ thống kiểu cơ bản}

[Nội dung Chapter 3 sẽ được viết tiếp...]

\chapter{Null Safety - Giải quyết lỗi tỷ đô}

[Nội dung Chapter 4 sẽ được viết tiếp...]

\chapter{Kết luận}

Báo cáo này đã trình bày tổng quan về ngôn ngữ lập trình Kotlin từ góc độ nguyên lý ngôn ngữ lập trình. Các chapters tiếp theo sẽ phân tích chi tiết các khía cạnh về type system, functional programming, và các tính năng nâng cao của Kotlin.

Trong nhiều quy trình sản xuất công nghiệp và dịch vụ, có nhiều công việc lặp đi lặp lại và đòi hỏi sức lao động lớn. Nhưng với sự phát triển của công nghệ trí tuệ nhân tạo, các hệ thống tự động hóa có khả năng thay thế nhân công trong những công việc đơn giản và lặp lại. Điều này giúp giảm thiểu sự phụ thuộc vào lao động, tiết kiệm chi phí và tăng hiệu suất sản xuất.
\begin{figure}[!ht] 
	\centering 
	\includegraphics[scale=0.2]{images/ai-trong-san-xuat-cong-nghiep.jpeg}
	\caption{Cánh tay robot thao tác chính xác sản phẩm } \label{fig:cce_vs_mae}	
% Ctrl + /
\end{figure}
\subsubsection{Nâng cao khả năng tương tác và trải nghiệm khách hàng}
Trong dịch vụ, ứng dụng của AI mang lại nhiều lợi ích đáng kể, đặc biệt là nâng cao trải nghiệm khách hàng, đặc biệt là Trợ lý ảo có thể hỗ trợ tức thì và tận tâm và chính xác trong các lĩnh vực như sau: 
\begin{itemize}
  \item Trợ lý ảo trong Dịch vụ Khách hàng:
Chatbot và trợ lý ảo có khả năng xử lý hàng loạt yêu cầu từ khách hàng một cách tự động và nhanh chóng.
Cung cấp thông tin về sản phẩm, chính sách, và hướng dẫn sử dụng mà không yêu cầu sự can thiệp của nhân viên.
  \item Hỗ trợ Tư vấn và Mua sắm:
Trợ lý ảo có thể tư vấn cho khách hàng về sản phẩm và dịch vụ dựa trên sở thích và nhu cầu cá nhân.
Hỗ trợ quá trình mua sắm trực tuyến bằng cách đưa ra các gợi ý sản phẩm và giải đáp thắc mắc.
  \item Giải quyết vấn đề và Hỗ trợ Kỹ thuật:
Trí tuệ nhân tạo có thể giải quyết một số vấn đề cơ bản của khách hàng mà không cần đến sự can thiệp của nhân viên.
Hỗ trợ kỹ thuật tức thì bằng cách cung cấp hướng dẫn sửa chữa, gỡ lỗi, hoặc đưa ra các giải pháp cơ bản.
   \item  Tương tác Ngôn ngữ Tự nhiên:
Trợ lý ảo có khả năng tương tác với khách hàng theo ngôn ngữ tự nhiên, giúp tạo ra một trải nghiệm tận tâm và thân thiện.
Hiểu và xử lý các yêu cầu, câu hỏi của khách hàng một cách linh hoạt và chính xác.
   \item Theo dõi và Phản hồi:
AI có thể tự động theo dõi hoạt động và thu thập phản hồi từ khách hàng, từ đó cung cấp dữ liệu quan trọng để cải thiện dịch vụ trong tương lai.
\end{itemize}

\begin{figure}[!ht] 
    \centering 
    \includegraphics[scale=0.7]{images/chat_bot.jpeg}
    \caption{ Chatbot tương tác với con người } %\label{fig:cce_vs_mae}	
% Ctrl + /
\end{figure}

\subsubsection{Sáng tạo nghệ thuật}
Đặc biệt trong thời kỳ các mô hình AI sinh càng phát triển, ứng dụng của AI trong sáng tạo nghệ thuật là rất đáng kể, cụ thể như sau: 
\begin{itemize}
   \item Tạo Hình Ảnh và Nghệ Thuật Đồ Họa:
AI có thể sáng tạo ra hình ảnh và nghệ thuật đồ họa mới thông qua GANs,Stable Diffusion, hai loại mô hình máy học của mạng sinh phổ biến nhất hiện nay. GANs và Stable Deffusion có khả năng tạo ra những hình ảnh độc đáo và rất chân thực, thậm chí con người không phân biệt được thật giả, dựa trên dữ liệu đào tạo mà chúng được cung cấp. Hai nền tảng mô sinh sinh ảnh từ mô tả là DALL-3 và Midjourney do công ty Microsoft và Midjourney đang làm mưa làm gió trong cộng đồng nghệ thời gian gần đây

\begin{figure}[!ht] 
    \centering 
    \includegraphics[scale=0.7]{images/A_mechanical_dove_8274822e-52fe-40fa-ac4d-f3cde5a332ae.png}
    \caption{ Hình ảnh một chú chim giả được sinh ra từ AI} %\label{fig:cce_vs_mae}	
% Ctrl + /
\end{figure}
\newpage

\item Soạn Nhạc và Âm Nhạc:
AI có thể tạo ra bản nhạc mới hoặc thậm chí học theo phong cách của các nghệ sĩ nổi tiếng. Các mô hình như Magenta của Google đã thực hiện nhiều dự án về sáng tác âm nhạc.


\item Viết và Sáng Tác Văn Bản:
Chatbots và Tạo Nội Dung: AI có thể được sử dụng để tạo nội dung văn bản, từ bài viết đến các đoạn trích trong sách. OpenAI đã phát triển mô hình GPT-3.5 và hiện nay là GPT-4, các công ty khác cũng ra mắt các mô hình tương tự như BARD của google, grok của xAI có khả năng tạo ra văn bản tự nhiên và chất lượng cao.


\end{itemize}
\subsection{Tăng cường khả năng giải quyết vấn đề}
Ngoài ra, AI còn giúp con người tăng khả năng giải quyết, hỗ trợ dự đoán lợi ích và tác hại nhiều vấn đề như :
\begin{itemize}
\item Chẩn Đoán Bệnh Lý trong Y Tế:
AI có khả năng phân tích hình ảnh y khoa như tia X, MRI, và CT để giúp bác sĩ đưa ra chẩn đoán chính xác và nhanh chóng. Điều này giúp tối ưu hóa thời gian và nguồn lực y tế.

\begin{figure}[!ht] 
    \centering 
    \includegraphics[scale=0.7]{images/ungthuPhoi.jpeg}
    \caption{Ứng dụng AI trong việc chẩn đoán ung thu phổi } %\label{fig:cce_vs_mae}	
% Ctrl + /
\end{figure}

\item Tự Động Hóa Công Việc Cơ Bản:
AI được sử dụng để tự động hóa các công việc lặp đi lặp lại, giúp tăng cường khả năng giải quyết vấn đề trong sản xuất và quản lý.

\item Dự Báo Thị Trường và Tổ Chức:
Các mô hình dự đoán AI sử dụng dữ liệu lớn để dự báo xu hướng thị trường và cung cấp thông tin hỗ trợ quyết định trong chiến lược kinh doanh.
\item Quản Lý Dữ Liệu Lớn:
AI giúp quản lý và phân tích dữ liệu lớn, tìm kiếm mối liên quan và xu hướng, hỗ trợ quyết định trong nghiên cứu và phát triển.

\item Tư Vấn Tài Chính và Đầu Tư:
Hệ thống tư vấn AI có khả năng dự đoán diễn biến thị trường tài chính, hỗ trợ quyết định đầu tư và quản lý rủi ro.

\item Quản Lý Chuỗi Cung Ứng:
AI được tích hợp vào quản lý chuỗi cung ứng để dự đoán và quản lý tồn kho, cung cấp thông tin về tình trạng sản xuất và giao hàng.


\item Dự Đoán Hỏa Hoạn và Thảm Họa Thiên Nhiên:
Hệ thống dự đoán AI sử dụng dữ liệu từ cảm biến và mô hình để cảnh báo và giảm thiểu thiệt hại do hỏa hoạn, lụt lội, và các thảm họa thiên nhiên khác.


\item Tối Ưu Hóa Giao Thông và Quy Hoạch Đô Thị:
AI được sử dụng để tối ưu hóa luồng giao thông, cải thiện quy hoạch đô thị thông minh và giảm kẹt xe.

\item Tạo Nội Dung và Tương Tác Người Máy:
AI có khả năng tạo ra nội dung sáng tạo, tương tác với người dùng qua trợ lý ảo, và cung cấp trải nghiệm người dùng cá nhân hóa.

\begin{figure}[!ht] 
    \centering 
    \includegraphics[scale=0.5]{images/ai-thoi-trang-2-3004.jpg.jpeg}
    \caption{ AI phân tích và gợi ý thời trang} %\label{fig:cce_vs_mae}	
% Ctrl + /
\end{figure}

\item Dự Đoán Hỏi Đáp và Hỗ Trợ Khách Hàng:
Các hệ thống AI dự đoán câu hỏi và cung cấp hỗ trợ khách hàng thông qua trò chuyện trực tuyến, giúp tăng cường khả năng giải quyết vấn đề trong dịch vụ khách hàng.



\end{itemize}










\chapter{Những ảnh hưởng tiêu cực của AI đối với con người}
% \bigbreak
Mặc dù trí tuệ nhân tạo (AI) có rất nhiều ưu điểm, rất nhiều hữu ích cho con người, và  đang phát triển với tốc độ chóng mặt, sức mạnh và tiềm năng của nó đang mở ra những cánh cửa mới đầy hứa hẹn cho sự tiến bộ và thay đổi trong mọi lĩnh vực của cuộc sống. Tuy nhiên, sự phổ cập và tích hợp rộng rãi của AI cũng mang theo những mối nguy hại không đáng coi nhẹ, tác động không chỉ đến cá nhân mà còn đến xã hội và tất cả cảnh quan của loài người. 
\begin{itemize}
\item  Chính giám đốc điều hành OpenAI công ty tạo ra ChatGPT rất nổi hiện nay là ông SAMUEL ALTMAN đã nói 'Nếu công nghệ AI này mà đi sai đường, nó sẽ đi rất sai' .
\item Geoffrey Hinton, người được mệnh danh là "bố già" của AI, thậm chí đã từ chức tại Google để công khai chỉ trích công nghệ do chính ông mở lối. Theo ông Geofrey Hinton, nguy cơ của AI còn khẩn cấp hơn biến đổi khí hậu. Người ta có thể đưa ra những khuyến nghị để ngăn biến đổi khí hậu, nhưng với AI không rõ nên làm gì. Và nhân loại sẽ chỉ là một giai đoạn trong quá trình tiến hóa của AI. 

\item Một "bố già" AI khác, Giáo sư Yoshua Bengio, Nhà sáng lập Viện nghiên cứu AI Mila cũng đã khẳng định quân đội không được phép sử dụng sức mạnh của AI vì đó là một trong những "nơi tồi tệ nhất có thể đặt AI siêu thông minh".

Tất cả đã cho thấy sự nguy hại thật sự của công nghệ AI tới loài người chúng ta.

Chương này tập trung đào sâu vào việc phân tích những mối nguy hại đáng chú ý của AI và cách chúng có thể ảnh hưởng đến loài người từ nhiều góc độ khác nhau. Từ những thách thức đạo đức và quản lý mà sự tiến triển của AI đặt ra, đến những ảnh hưởng lớn đối với thị trường lao động và cuộc sống hàng ngày, chúng ta sẽ đặt ra những câu hỏi cần thiết về hướng đi của chúng ta khi tiếp tục khám phá khả năng của trí tuệ nhân tạo. 

\end{itemize}

% \section{Mất Việc Làm và Thay Thế Người Lao Động}
\section{Ảnh Hưởng Đến Một Số Ngành Nghề}
Trong sự trỗi dậy của AI, việc tự động hóa ngày càng mở ra khả năng thay thế người lao động trong nhiều ngành nghề. Các hệ thống tự động và robot có khả năng thực hiện công việc một cách hiệu quả và không cần nghỉ ngơi, gây áp lực lớn lên người lao động truyền thống.

Các ngành nghề như sản xuất, dịch vụ khách hàng, và giao thông vận tải đang phải đối mặt với thách thức nghiêm trọng khi mà công nghệ AI có khả năng thực hiện các nhiệm vụ công việc một cách tự động. Những công việc lặp đi lặp lại và có tính chất dự đoán cao có nguy cơ bị thay thế, đặt ra thách thức đối với sự ổn định của nguồn việc làm truyền thống.

Điều này không chỉ ảnh hưởng đến người lao động ở cấp độ cá nhân mà còn có thể gây ra sự biến động và không ổn định trong thị trường lao động, đặt ra câu hỏi về cách xã hội phản ứng và điều chỉnh để bảo vệ quyền lợi và sự công bằng trong môi trường làm việc mới, do ảnh hưởng của AI.

Theo báo cáo của Goldman Sachs (Mỹ), 300 triệu việc làm có nguy cơ bị thay thế bởi trí tuệ nhân tạo (AI) như ChatGPT và các giải pháp công nghệ khác. Báo cáo cho rằng AI có thể tự động hóa 25\% công việc của toàn bộ thị trường lao động. Những công việc như quản trị, pháp lý và kiến trúc là 3 nhóm ngành có tỉ lệ thay thế bởi AI cao, theo thứ tự lên đến 46\%, 44\% và 37\%. Dữ liệu do TechRadar Pro (Mỹ) thu thập cũng cho thấy những công việc có nguy cơ bị thay thế bởi AI đã tăng đáng kể từ tháng 11-2022, khi ChatGPT được giới thiệu đến công chúng. Nhiều chuyên gia công nghệ cho rằng 2023 là năm bùng nổ của AI khi 1 trong 6 xu hướng công nghệ được đầu tư mạnh mẽ nhất từ trước đến nay. Theo báo cáo "Chỉ số sẵn sàng AI của chính phủ năm 2022" do Tổ chức Oxford Insights (Anh) thực hiện hồi tháng 12-2022, Việt Nam hiện xếp hạng 55/181 quốc gia tham gia khảo sát và xếp thứ 6/10 trong khu vực ASEAN. Kết quả này không chỉ phản ánh rõ nét sự phát triển của công nghệ AI, mà còn cho thấy xu hướng hình thành nền công nghiệp AI tại Việt Nam. 

Từ những thông tin trên, chúng ta có thể khẳng định rằng trong tương lai, một số ngành nghề nhất định, có tính lặp lại có thể được thay thế một phần hoặc hoàn toàn bởi AI. Điều này cũng là một phần tạo nên sự phát triển của xã hội.Tuy nhiên câu hỏi đặt ra với chúng ta là khi một sản phẩm do AI tạo nên bị lỗi, gây hậu quả khôn lường tới người sử dụng thì ai sẽ là người phải chịu trách nhiệm ?


\section{Những Thách Thức Xã Hội và Kinh Tế}

Trong bối cảnh mà AI ngày càng thâm nhập sâu rộng vào các lĩnh vực kinh tế, xuất hiện những thách thức xã hội và kinh tế đáng chú ý. Mất việc làm và sự thay thế người lao động bởi máy móc và trí tuệ nhân tạo không chỉ ảnh hưởng đến cá nhân mà còn tạo ra những tác động đáng kể trên cấp độ toàn cầu.

\begin{itemize}

\item Tăng Cường Chia Rẽ Xã Hội:
Sự chênh lệch giữa những người có kỹ năng phù hợp với công nghệ mới và những người mất việc có thể dẫn đến sự gia tăng chia rẽ xã hội. Những người không có kỹ năng cần thiết có thể gặp khó khăn trong việc thích ứng với môi trường lao động mới, tạo ra sự bất bình và không hài lòng xã hội.

\item Thách Thức Đối Mặt với Cộng Đồng Lao Động:
Cộng đồng lao động có thể phải đối mặt với thách thức lớn khi cần chuyển đổi nghề nghiệp hoặc học lại kỹ năng mới để thích ứng với môi trường làm việc mới. Việc này đặt ra nhu cầu cần có hệ thống đào tạo và tái đào tạo mạnh mẽ để hỗ trợ họ.

\item Ảnh Hưởng Đến Kinh Tế Quốc Gia:
Mất việc làm và sự thay thế người lao động có thể tạo ra áp lực lớn đối với kinh tế quốc gia. Giảm nguồn thu nhập và tăng lên chi phí xã hội có thể ảnh hưởng đến các chính sách xã hội và y tế, đặt ra thách thức lớn đối với sự ổn định kinh tế.

\item Cần Thiết Sự Điều Chỉnh và Chính Sách Quản Lý:
Để đối mặt với những thách thức này, cần có sự điều chỉnh linh hoạt trong chính sách quản lý và chính sách xã hội. Việc phát triển các biện pháp hỗ trợ người lao động, tái đào tạo nghề nghiệp và chính sách phân phối lại thu nhập trở nên quan trọng để giảm bớt tác động tiêu cực của sự thay đổi do AI.

\end{itemize}

\section{Lo ngại về độ tin cậy và an toàn của hệ thống AI}
\subsubsection{Nguy Cơ về Quyết Định Tự Động}
Trong thế giới đầy ứng dụng của trí tuệ nhân tạo, nguy cơ liên quan đến quyết định tự động của hệ thống AI trở nên ngày càng đáng quan ngại. Các thuật toán máy học và học sâu có thể đưa ra quyết định quan trọng trong nhiều lĩnh vực như y tế, tài chính, và an ninh mà không có sự can thiệp trực tiếp của con người.

\begin{itemize}

\item  Thiếu Giải Thích Rõ Ràng:
Một trong những thách thức lớn là khả năng giải thích rõ ràng về cách các hệ thống AI đưa ra quyết định. Nhiều thuật toán học sâu hoạt động như "hộp đen," làm tăng khả năng xâm phạm quyền riêng tư và giảm khả năng theo dõi và kiểm soát của con người.



\item Nguy Cơ Về Quyết Định Thiên Vị:
Hệ thống AI có thể tỏ ra thiên lệch do dữ liệu huấn luyện không đại diện cho đầy đủ đa dạng trong xã hội. Điều này có thể dẫn đến quyết định không công bằng và ảnh hưởng tiêu cực đến các nhóm dân tộc, giới tính, hoặc địa lý cụ thể. Công hệ AI cũng có thể tạo ra phân biệt đối xử và vi phạm quyền được tôn trọng sự khác biệt của mỗi cá nhân nếu dữ liệu “huấn luyện” cho AI trong quá trình “học máy” (machine learning) là không đầy đủ hoặc bị điều chỉnh theo hướng thiên vị một nhóm nào đó một cách có chủ đích. Ngày càng có nhiều bằng chứng cho thấy người da đen, phụ nữ, dân tộc thiểu số, người khuyết tật và đặc biệt là nhóm LGBT bị phân biệt đối xử bởi các thuật toán AI. Tháng 5-2022, một nghiên cứu của Cơ quan Quyền cơ bản của Liên minh châu Âu đã đưa ra kết luận AI có thể làm gia tăng phân biệt đối xử. Gần đây, Tuyên bố Toronto cũng kêu gọi biện pháp ngăn chặn các hệ thống học máy làm trầm trọng hơn tình trạng phân biệt đối xử trong xã hội hiện đại.

\item Khả Năng Tự Học và Tự Thích Ứng:
Một số hệ thống AI có khả năng tự học và tự thích ứng theo thời gian mà không có sự can thiệp của con người. Điều này tạo ra nguy cơ không kiểm soát được khi hệ thống tự thay đổi và đưa ra quyết định mà không có sự hiểu biết hoặc kiểm soát của con người.

\item Quyết Định Dựa Trên Dữ Liệu Không Chính Xác:
Nếu dữ liệu đầu vào không chính xác hoặc không đại diện, hệ thống AI có thể đưa ra quyết định sai lầm. Trong trường hợp các ứng dụng như y tế hoặc an ninh, quyết định không chính xác có thể mang lại hậu quả nghiêm trọng.

\item Rủi Ro Tăng Cường Khác Như Đưa Thông Tin Giả:
Các hệ thống AI được tích hợp vào các lĩnh vực như quản lý giao thông, tài chính, và y tế đang tăng cường quyết định và tương tác trong thế giới thực. Nếu không kiểm soát chặt chẽ, sự tăng cường này có thể tạo ra rủi ro và hậu quả không dự đoán được.
\end{itemize}

\subsubsection{Hậu Quả Của Lỗi Mà Không Có Kiểm Soát Con Người}


Trong quá trình phát triển và triển khai hệ thống trí tuệ nhân tạo (AI), nguy cơ của việc xảy ra lỗi mà không có sự kiểm soát của con người có thể mang lại hậu quả nghiêm trọng, đặt ra những thách thức lớn đối với độ tin cậy và an toàn:
\begin{itemize}

\item Ảnh Hưởng Đến An Toàn và Sức Khỏe Công Dân:
Trong các lĩnh vực như y tế, một quyết định sai lầm của hệ thống AI có thể ảnh hưởng trực tiếp đến an toàn và sức khỏe của bệnh nhân. Việc chấp nhận quyết định của hệ thống mà không kiểm soát có thể tạo ra rủi ro lớn.


\item Hậu Quả Pháp Lý và Đạo Đức:
Những lỗi từ hệ thống AI không chỉ mang lại hậu quả pháp lý mà còn đặt ra những câu hỏi lớn về trách nhiệm đạo đức. Ai chịu trách nhiệm khi có sự cố xảy ra do quyết định của máy móc mà không có sự can thiệp của con người?


\item Tác Động Đến Niềm Tin Công Dân:
Các sự cố và lỗi của hệ thống AI có thể ảnh hưởng đến niềm tin của công dân vào công nghệ và quyết định tự động. Việc này có thể dẫn đến sự phản đối và hoài nghi, ảnh hưởng đến sự chấp nhận và tích cực hóa của công dân đối với sự phổ cập của AI trong xã hội.

\item Khả Năng Xâm Phạm Quyền Riêng Tư:
Trong trường hợp lỗi, thông tin cá nhân có thể bị xâm phạm và lộ ra ngoại trừ, đặt ra vấn đề lớn về quyền riêng tư và an ninh thông tin của người sử dụng.






\end{itemize}



% \chapter{Tác động của AI đến xã hội}
\section{Tác động tới tâm lý con người} 

Sự phổ cập rộng rãi của trí tuệ nhân tạo mà cụ thể là các mô hình trợ lý ảo có thể dẫn đến việc giảm tương tác xã hội trực tiếp giữa con người và con người. Sự giảm tương tác trực tiếp có thể dẫn đến sự giảm thiểu kết nối  người và người. Gặp gỡ và tương tác trực tiếp giữa con người mang lại sự ấm áp và đồng cảm, điều mà trí tuệ nhân tạo không thể thay thế hoàn toàn. Đặc biệt, với sự giảm tương tác trực tiếp, có thể xuất hiện thách thức đối mặt với kỹ năng xã hội. Nhất là đối với thế hệ trẻ lớn lên trong môi trường nền tảng trực tuyến, việc thiếu kinh nghiệm tương tác trực tiếp có thể ảnh hưởng đến khả năng giao tiếp và tương tác xã hội. 
Thậm chí hiện nay, chúng ta có thể lựa chọn bạn trai, bạn gái ảo với nhiều phong cách với các mô hình Nhân vật ảo.

\begin{figure}[!ht] 
    \centering 
    \includegraphics[scale=0.7]{images/AI_girl.jpeg}
    \caption{Hình ảnh bạn gái ảo } %\label{fig:cce_vs_mae}	
% Ctrl + /
\end{figure}

Điều này dẫn đến hậu quả khôn lường về mặt tâm sinh lý dẫn đến kết hôn muộn hoặc không kết hôn tăng cao.

\section{Nguy cơ bị lợi dụng tạo thông tin giả và lừa đảo} 
Sự phổ cập của trí tuệ nhân tạo có thể tăng nguy cơ về thông tin giả mạo và đánh lừa. Công nghệ AI có khả năng tạo ra nội dung giả mạo chân thực, từ tin tức đến hình ảnh và video, làm tăng nguy cơ người dùng bị lừa đảo. 

Một dẫn chứng cụ thể, đó là vào 22/5/2023, một tin tức giả mạo được tạo từ AI về vụ nổ tại Nhà Trắng, đã được lan truyền một cách nhanh chóng.
\begin{figure}[ht] 
    \centering 
    \includegraphics[scale=0.7]{images/hinh-anh-gia-ve-vu-no-gan-lau-nam-goc-khien-thi-truong-chung-khoan-my-giam-diem.jpeg}
    \caption{Hình ảnh giả mạo vụ nổ tại Nhà Trắng Mỹ, được tạo bởi AI } %\label{fig:cce_vs_mae}	
% Ctrl + /
\end{figure}

Hậu quả đó là sự lan truyền của bức ảnh còn khiến thị trường chứng khoán Mỹ lao đao trong một khoảng thời gian ngắn ngay sau khi mở cửa giao dịch.
Theo đó, chỉ số S\&P 500 trên thị trường chứng khoán Mỹ đã giảm khoảng 0,3\%, xuống thấp nhất trong phiên, khi các tài khoản mạng xã hội và trang web đầu tư lan truyền các thông tin sai lệch về vụ nổ giả. Ngược lại, giá trái phiếu kho bạc Mỹ và giá vàng bắt đầu tăng nhanh cho thấy các nhà đầu tư đang tìm kiếm một nơi an toàn hơn để dự trự tài sản.

Deepfake là công nghệ được biết đến nhiều nhất với tác dụng tái tạo lại khuôn mặt của người trong video nhờ sử dụng trí tuệ nhân tạo (AI).
Ban đầu, công nghệ này được sinh ra cho mục đích giải trí, giúp người dùng lồng khuôn mặt, giọng nói của mình vào các nhân vật yêu thích trên video, mà vẫn đảm bảo hoạt động giống như được quay thực tế. Giới tội phạm nhanh chóng lợi dụng điều đó để tạo ra công cụ giả mạo người khác, giúp chúng thực hiện các vụ lừa đảo, hoặc lan truyền tin thất thiệt trên mạng.
Không chỉ trên thế giới mà ngay ở Việt Nam, sức mạnh của trí tuệ nhân tạo AI đang bị kẻ gian lợi dụng làm công cụ lừa đảo. Đây không chỉ là thủ đoạn mới mà còn cực kỳ tinh vi khiến nạn nhân dễ dàng sập bẫy. Thời gian gần đây trên không gian mạng xuất hiện một phương thức lừa đảo chiếm đoạt tài sản mới thông qua các tài khoản mạng xã hội. Các đối tượng đã sử dụng công nghệ Deepfake, công nghệ ứng dụng trí tuệ nhân tạo AI để tạo ra các sản phẩm công nghệ có âm thanh, hình ảnh và video giả theo giọng nói, khuôn mặt con người với độ chính xác rất cao. Mặc dù đã rất cảnh giác, nhưng đã có những nạn nhân mất hàng chục, thậm chí cả trăm triệu đồng bởi những kẻ lừa đảo.

\begin{figure}[ht] 
    \centering 
    \includegraphics[scale=0.8]{images/loso_AI.jpeg}
    \caption{Minh hoạ đối tượng lợi dụng AI để lừa đảo } %\label{fig:cce_vs_mae}	
% Ctrl + /
\end{figure}

Do đó, Cục điều tra Liên bang (FBI) vừa đưa ra một lời cảnh báo về việc tội phạm mạng và tin tặc đang tận dụng các chương trình A.I để tạo ra phần mềm độc hại và lên kịch bản cho những hành vi lừa đảo trực tuyến.Dựa vào những đoạn mã do A.I viết ra, tin tặc có thể tinh chỉnh để thêm các chức năng cho các loại mã độc. Điều này cho phép tin tặc có thể rút ngắn thời gian xây dựng các loại mã độc, giúp chúng có thể qua mặt được các phần mềm bảo mật để xâm nhập vào máy tính của nạn nhân. 
Ngoài ra, tin tặc còn có thể nhờ phần mềm A.I xây dựng các trang web một cách nhanh chóng để phát tán mã độc lên internet.
Tội phạm mạng còn lợi dụng A.I giúp tạo những hình ảnh hoặc giọng nói giả mạo để thực hiện các cuộc gọi lừa đảo người thân của họ nhằm lấy cắp tiền. FBI cho biết các phần mềm A.I còn bị tội phạm công nghệ cao lợi dụng để ghép mặt người khác vào những hình ảnh và video khiêu dâm, từ đó tống tiền các nạn nhân.

FBI nhận định rằng các hình thức tấn công mạng và lừa đảo trực tuyến lợi dụng A.I là vấn đề gây ảnh hưởng đến an ninh quốc gia và đang tìm giải pháp để xử lý vấn đề, bao gồm đề nghị các công ty phát triển A.I hợp tác để có giải pháp nhận diện những nội dung giả mạo được tạo ra bởi A.I.

Đây không phải là lần đầu tiên các chuyên gia cảnh báo về việc A.I bị tội phạm mạng và tin tặc lợi dụng để thiết kế mã độc.
Đầu năm nay, các chuyên gia của hãng nghiên cứu bảo mật Check Point (Israel) đã lên tiếng cảnh báo về việc tin tặc lợi dụng ChatGPT, một trong những phần mềm trí tuệ nhân tạo được đánh giá là thông minh nhất hiện nay, để phát triển mã độc và các công cụ hack.
Check Point cho biết một tin tặc đã chia sẻ lên diễn đàn nổi tiếng dành cho hacker loại mã độc nhắm đến nền tảng Android, mà theo hacker này được lập trình bởi ChatGPT. Mã độc này có khả năng lây nhiễm lên smartphone chạy Android để lấy cắp dữ liệu.
Hacker này cũng đã trình diễn một mã độc khác được viết bởi ChatGPT, có khả năng xâm nhập vào máy tính để mở cửa hậu, cho phép tin tặc có thể âm thầm xâm nhập vào máy tính hoặc cài đặt thêm các loại mã độc khác nhau lên thiết bị.
Cảnh báo của FBI và Check Point đã khiến nhiều người lo ngại, khi viễn cảnh trí tuệ nhân tạo bị lợi dụng và trở thành tay sai của kẻ xấu giống như trong các bộ phim khoa học viễn tưởng hoàn toàn có thể xảy ra.







\section{Nguy cơ tác động trực tiếp tới sinh mạng con người khi AI được áp dụng trong quân sự}
\subsection{Cuộc đua áp dụng AI trong quân sự}
Giáo sư Yoshua Bengio, Nhà sáng lập Viện nghiên cứu AI Mila cũng đã khẳng định quân đội không được phép sử dụng sức mạnh của AI vì đó là một trong những "nơi tồi tệ nhất có thể đặt AI siêu thông minh". Tuy nhiên, phớt lờ cảnh báo này, cuộc chạy đua ứng dụng trí tuệ nhân tạo vào quân sự đang được thúc đẩy mạnh ở tất cả các quốc gia.
Trong bài viết trên tạp chí Foreign Policy, cây viết Michael Hirsh nhận định: "AI đang cách mạng hóa hoạt động tác chiến tương lai". Hiện có ít nhất 50 quốc gia đang nghiên cứu robot chiến trường, tích hợp AI vào các khí tài quân sự, trong đó cuộc đua của 3 "ông lớn"  Nga - Mỹ - Trung đang diễn ra tương đối quyết liệt.
Cả 3 nước đều đang dồn các nguồn lực vào việc phát triển công nghệ AI trong vũ khí.

Trong tương lai, các cuộc chiến có thể chỉ diễn ra giữa các đoàn quân robot, điều này giảm thiểu sự hi sinh của người lính, nhưng cũng mang lại hậu quả khôn lường, khi Trí tuệ nhân tạo nắm trong tay công cụ có thể tác động trực tiếp tới tính mạng con người.

\begin{figure}[ht] 
    \centering 
    \includegraphics[scale=0.7]{images/uav.jpg}
    \caption{Hình ảnh mô phỏng một dự án máy bay không người lái tấn công kiểu bầy đàn của Mỹ } %\label{fig:cce_vs_mae}	
% Ctrl + /
\end{figure}

Bộ Quốc phòng Mỹ đang triển khai kế hoạch đẩy mạnh ứng dụng AI trong lĩnh vực không gian thông qua chương trình bí mật của không quân Mỹ có tên gọi “Chiếm ưu thế trên không thế hệ tiếp theo” (NGAD). Khi hoàn thành, dự án sẽ hỗ trợ cho khoảng 1.000 UAV phối hợp tác chiến cùng 200 máy bay chiến đấu thông thường.

Tuy nhiên, việc sử dụng AI tiềm ẩn rủi ro cao hơn vì có thể khiến các cường quốc rút ngắn thời gian ra quyết định chiến lược xuống vài phút thay vì hàng giờ hoặc vài ngày như trước đây, do việc ra quyết định trong các tình huống khó khăn đối với con người cần phải căn cứ vào rất nhiều thông tin, các đánh giá chiến lược và chiến thuật. Thế nhưng, với sự hỗ trợ của AI, quá trình ra quyết định rút ngắn, thậm chí ngay cả trong các tình huống phải ra quyết định sống còn và nếu chiến tranh hạt nhân nổ ra.

Herbert Lin, một chuyên gia AI của Đại học Stanford cho biết, điều nguy hiểm là những người ra quyết định có thể chỉ cần dựa vào các thông tin do AI cung cấp rồi đưa ra các mệnh lệnh như một phần của hệ thống chỉ huy và kiểm soát vũ khí, kể cả vũ khí hạt nhân. Họ tin rằng, AI có khả năng xử lý, hoạt động với tốc độ nhanh hơn rất nhiều so với con người mà không hề bị chi phối bởi yếu tố khác như tình cảm, tính nhân đạo.


\subsection{Nguy cơ mất an ninh}

Trí tuệ nhân tạo có thể trở thành mối nguy đối với an ninh quốc gia và an toàn cá nhân. Elon Musk đã cảnh báo rằng AI có thể trở thành một công cụ đáng sợ trong chiến tranh. Nếu một quốc gia có thể phát triển một hệ thống trí tuệ nhân tạo vượt trội, quốc gia đó có thể sử dụng nó để tấn công các quốc gia khác. Giáo sư khoa học máy tính Stuart Russell tại Đại học California là người từng dành hàng chục năm nghiên cứu về trí tuệ nhân tạo. Cảnh báo về mối hiểm họa đối với an ninh do AI mang lại, ông cho rằng hiện nay chúng ta đã có thể sản xuất vũ khí tấn công tự động bằng cách tích hợp và thu nhỏ những công nghệ sẵn có.

Thí nghiệm được Giáo sư Stuart Russell cùng Viện nghiên cứu Cuộc sống tương lai (FLI) tiến hành sử dụng một robot sát thủ (slaughterbot) là thiết bị bay siêu nhỏ, trang bị camera, cảm biến, phần mềm xử lý hình ảnh, nhận diện khuôn mặt, một khối thuốc nổ nặng 3gr, và một bộ vi xử lý tốc độ cao, cho phép phân tích dữ liệu và phản ứng nhanh hơn 100 lần tốc độ bộ não con người. Theo thông số lập trình, robot sát thủ truy cập liên tục vào dữ liệu đám mây để tìm kiếm thông tin về mục tiêu và tìm cách tiếp cận. Khi đã tìm được, nó sẽ lao thẳng vào mục tiêu với vận tốc lớn, kích hoạt khối thuốc nổ 3gr, khoan sâu vào bên trong hộp sọ, giết chết nạn nhân trong nháy mắt.

\begin{figure}[ht] 
    \centering 
    \includegraphics[scale=0.18]{images/china-blowfish-a3-weapon-drone.jpeg}
    \caption{Minh hoạ drone áp dụng AI } %\label{fig:cce_vs_mae}	
% Ctrl + /
\end{figure}

Đó chỉ là một thí nghiệm với AI ở cấp độ đơn giản nhất. Nếu AI được sử dụng để phát triển các phương thức tấn công mới, tinh vi hơn, nó cũng có thể làm tăng khả năng tấn công của nhóm những kẻ tấn công và dẫn đến các hậu quả nghiêm trọng hơn so với các cuộc tấn công thông thường. Khi AI được phát triển đến trình độ có thể tự ra quyết định để đối phó với sự biến đổi của môi trường xung quanh hoặc tự tìm kiếm mục tiêu thay thế, hay mở rộng phạm vi mục tiêu thì có lẽ loài người sẽ không còn an toàn nữa. Rất nhiều đồng nghiệp trong lĩnh vực này đồng ý với Giáo sư Stuart Russell và cho rằng vẫn còn có cơ hội để ngăn chặn một tương lai ảm đạm như vậy, nhưng chúng ta cũng không thực sự còn nhiều thời gian.

\subsection{Nguy cơ AI trở thành siêu AI, có thể tự chủ và kiểm soát con người}

AI có thể được thiết kế sai lệch hoặc được “huấn luyện” (training) thông qua cơ chế học máy (machine learning) không đúng cách và có thể thực hiện những hành động không mong muốn, gây thiệt hại cho con người và môi trường. Ví dụ cụ thể như sau: Với cơ chế học máy, AI ngày càng trở nên thông minh hơn. Khi trí tuệ nhân tạo đạt mức độ thông minh mà từ chỗ hỗ trợ các hệ thống quan trọng như điện lực, giao thông, y tế, tài chính..., nó có thể làm chủ và kiểm soát toàn bộ các hệ thống này và tự ra quyết định, thực hiện các quyết định đó trong các tình huống khẩn cấp. Khi AI được “trang bị” thêm các mục đích mang tính “dã tâm” có chủ đích (thiết kế sai lệch), nó có thể gây ra nhiều hậu quả nghiêm trọng như làm rối loạn hệ thống giao thông khi vô hiệu hóa hệ thống đèn giao thông hay ngắt điện hệ thống vận hành tàu điện đô thị, gây tai nạn liên hoàn, làm mất điện trên diện rộng...

Đã có khá nhiều bộ phim Hollywood và của nhiều nước được xây dựng theo kịch bản này. Tuy nhiên, với công nghệ AI hiện nay, điều này không còn là viễn cảnh quá xa xôi mà hoàn toàn có thể trở thành hiện thực. Elon Musk tin rằng nếu để AI phát triển không kiểm soát, đến mức nó có thể tự động hóa các quyết định mà không cần sự can thiệp của con người thì đây có thể là một nguy cơ cho sự tồn vong của con người. Đó là lý do ông cùng hàng nghìn chuyên gia công nghệ ký vào bức thư yêu cầu tạm ngừng và kiểm soát chặt chẽ quá trình phát triển AI một cách minh bạch. Theo Elon Musk, các hệ thống trí tuệ nhân tạo rất phức tạp, khó hiểu và việc kiểm soát chúng là rất khó khăn. Nếu thiếu đi sự minh bạch, việc sử dụng trí tuệ nhân tạo cho các mục đích thiếu đạo đức, gây thiệt hại cho con người chắc chắn sẽ xảy ra.



Sẽ như thế nào và ra sao, nếu các cỗ máy, tên lửa kia, nhận diện nhầm đối tượng xấu, nhận diện nhầm khu vực cần tấn, công. Hậu quả quá đã quá rõ ràng. Thậm chí, nếu như đội quân robot kia mất kiểm soát, không còn tuần lệnh con người, tấn công con người thì sao. Lúc đó, viễn cảnh trong bộ phim 'Kẻ Huỷ Diệt'  ngày diệt vong của loài người thực sự đã đến.
 

\section{Con người bị thao túng suy nghĩ, thậm chí bị điều khiển từ AI}
Hiện nay, từ khi ChatGPT ra đời, gần như chúng ta đã thay đổi thói quen tìm kiếm bằng Google hay các công cụ tương tự, bằng cách hỏi đáp qua ChatGPT, điều này hoàn toàn dẫn đến chúng ta có thể bị thao túng bởi những thông tin sai lệch từ AI. Ngoài ra, nếu sử nội dung sáng tạo bởi AI thì chỉ sau một thời gian, các tác phẩm, nội dung sẽ tràn ngập bởi AI, con người mất đi khả năng tư duy sáng tạo. Tuy nhiên, còn một hiểm hoạ ghê gớm hơn đó là 'AI có thể điều khiển hoàn toàn hành vi, cử chỉ của con người'. Viễn cảnh tưởng chừng như trong phim viễn tưởng này lại hoàn toàn có thể xảy ra bởi các minh chứng và nghiên cứu đột phá gần đây.

\subsection{Ghép chip vào não người}

Gắn chíp vào não người, chuyện tưởng rằng như phim viễn tưởng thì giờ đây hoàn toàn khả thi.Mới đây cơ quan quản lý dược phẩm và thực phẩm (FDA) của Mỹ mới đây đã cấp phép cho công ty khởi nghiệp Neuralink thử nghiệm cấy thiết bị này vào não người sau nhiều lần từ chối cấp phép. Công ty Neuralink của tỉ phú Elon Musk công ty thiết bị y tế của ông sẽ sớm bắt đầu các cuộc thử nghiệm cấy ghép não mang tính cách mạng để điều trị các bệnh nan y như bại liệt và mù loà...Nhưng những tranh cãi đã nổ ra từ đây. Đây sẽ là một bước tiến của khoa học hay sẽ làm một thảm họa, khi quyền riêng tư của con người bị xâm phạm và lợi dụng.

Những thử nghiệm gắn chíp vào não động vật như lợn và khỉ đã được công ty Neuralink thử nghiệm trước đây. Sau khi gắn chip vào não, chú khỉ có thể chơi video, hay gõ tay trên bàn phím máy tính và di chuyển con trỏ trên màn hình. Chú lợn có thể cử động chân theo sự điều khiển của các nhà khoa học. 

Con chip với đường kính chỉ 4 mm kết nối với bộ não thông qua hàng nghìn dây thần kinh nhân tạo siêu nhỏ được cấy vào não thông qua kỹ thuật khoan và cấy vào hộp sọ. Mục đích của công nghệ này theo Elon Mush là phục vụ cho y học. Việc giúp cấy chíp vào não bộ người có thể giúp chữa khỏi nhiều căn bệnh trong đó có béo phì, tự kỉ, trầm cảm, tâm thần phân liệt, Alzheimer, Parkinson, các bệnh về não, cột sống giao tiếp tốt hơn. Một ứng dụng khác của con chip là giúp con người sao lưu và mở lại ký ức vào một cơ thể mới, hay cơ thể robot. Trong thí nghiệm với khỉ, các nhà khoa học đã đưa dữ liệu hình ảnh quay bằng camera vào vỏ não thị giác của con khỉ, cho con vật thấy những tia sáng ảo khiến nó nghĩ bản thân đang ở các nơi khác nhau. Công nghệ của Neuralink đặc biệt có thể giúp người mù thấy được nhiều hình ảnh. Ở thí nghiệm với lợn, các nhà khoa học sử dụng điện cực ở tuỷ sống lợn để điều khiển các chuyển động chân khác nhau, công nghệ có thể giúp các người bị liệt tứ chi, đi lại hoặc sử dụng tay. Phương pháp của Neuralink không bao gồm chặn lệnh chuyển động của bộ não và chuyển hướng tới chân mà còn nghe tín hiệu cảm giác từ các chi và truyền ngược lại não bộ. Hiện tại công nghệ này vẫn còn trải qua chặng đường dài trước khi ứng dụng trong y học và cuối cùng là "giao tiếp với trí tuệ nhân tạo" cung cấp cho con người một bộ óc siêu phàm. 

\begin{figure}[ht] 
    \centering 
    \includegraphics[scale=1]{images/chip.jpeg}
    \caption{Bộ óc siêu phàm khi được gắn chip } %\label{fig:cce_vs_mae}	
% Ctrl + /
\end{figure}




\chapter{Các hành động để giảm thiểu, ngăn chặn các mối nguy hại từ AI}

Các nhà khoa học và giới chính trị gia các nước cũng đã nhìn nhận được sự nguy hiểm tiềm tàng của AI đến sự phát triển tự nhiên và cả sự tồn vong của loài người. Vì vậy, các cường quốc về trí tuệ nhân tạo đã có những động thái nhằm kiểm soát rủi ro, thách thức mà công nghệ này tạo ra.


\section{Kêu gọi ngừng phát triển hệ thống AI quá mạnh}

\section{HĐBA LHQ lần đầu tiên tổ chức thảo luận về nguy cơ từ trí tuệ nhân tạo}

\begin{figure}[ht] 
    \centering 
    \includegraphics[scale=0.5]{images/lhq.jpg}
    \caption{Toàn cảnh cuộc họp HĐBA LHQ tại New York, Mỹ, ngày 14/7/2023 } %\label{fig:cce_vs_mae}	
% Ctrl + /
\end{figure}



\section{UNESCO kêu gọi quản lý việc dùng AI trong trường học}


\section{Hội nghị thượng đỉnh cấp cao toàn cầu về an toàn AI}

\begin{figure}[ht] 
    \centering 
    \includegraphics[scale=0.3]{images/AI-safetysafety.jpeg}
    \caption{Thông điệp của hội nghị thượng đỉnh AI SAFETY SUMMIT } %\label{fig:cce_vs_mae}	
% Ctrl + /
\end{figure}

\subsubsection{Hợp tác để giám sát trí tuệ nhân tạo} 




\subsection{Pháp, Đức và Italy đạt thoả thuận về quản lý trí tuệ nhân tạo trong tương lai}



\subsection{Mỹ và Trung Quốc thỏa thuận về nghiêm cấm ứng dụng AI trong quân sự}

Tổng thống Mỹ Joe Biden và Chủ tịch Trung Quốc Tập Cận Bình dự kiến sẽ công bố một thỏa thuận lịch sử nhằm cấm ứng dụng trí tuệ nhân tạo (AI) trong lĩnh vực quân sự.

\begin{figure}[ht] 
    \centering 
    \includegraphics[scale=0.4]{images/ai_quansu.jpg}
    \caption{Cấm ứng dụng AI trong lĩnh vực quân sự có thể sẽ là thỏa thuận mang tính lịch sử giữa Mỹ và Trung Quốc } %\label{fig:cce_vs_mae}	
% Ctrl + /
\end{figure}




% \section{Đánh giá kết quả thực nghiệm}
\label{subsec:exp_evaludation}






\newpage
% \printbibliography[heading=bibintoc, title=Tài liệu tham khảo]
\chapter{Kết luận}


\newpage
\Huge{\textbf{Tài liệu tham khảo}}


\normalsize{[1]  https://vtv.vn/the-gioi/duc-phap-va-italy-dat-thoa-thuan-ve-quan-ly-tri-tue-nhan-tao-trong-tuong-lai-20231120072756083.htm}

\normalsize{[2]  https://dantri.com.vn/tam-diem/viet-nam-truoc-lan-song-tri-tue-nhan-tao-20231115141049147.htm}

\normalsize{[3] https://nhandan.vn/unesco-keu-goi-quan-ly-viec-su-dung-tri-tue-nhan-tao-ai-trong-truong-hoc-post771991.html}

\normalsize{[4]  https://nguoiquansat.vn/my-va-trung-quoc-se-bat-tay-cam-ung-dung-ai-trong-linh-vuc-quan-su-99227.html}

\normalsize{[5] https://vnexpress.net/chuyen-gia-lo-ngai-hiem-hoa-tu-ai-4615707.html}

\normalsize{[6] https://vnanet.vn/vi/graphic/nghe-thuat-van-hoa-va-giai-tri-1/hoi-nghi-thuong-dinh-ai-nhan-manh-yeu-cau-hop-tac-dam-bao-phat-trien-cong-nghe-an-toan-7062860.html}
% \addcontentsline{toc}{chapter}{Tài liệu tham khảo}
% \renewcommand\refname{Tài liệu tham khảo}

% \bibliographystyle{plain}
% \bibliography{references}

\end{document}
